\documentclass[lang=en,12pt,twoside]{textbook}
% ----------------------------- Math Font ------------------------------------- %
\usepackage[T1]{fontenc}
\usepackage{newtx,anyfontsize}
\DeclareSymbolFont{CMlargesymbols}{OMX}{cmex}{m}{n}
\let\sum\relax
\let\prod\relax
\DeclareMathSymbol{\sum}{\mathop}{CMlargesymbols}{"50}
\DeclareMathSymbol{\prod}{\mathop}{CMlargesymbols}{"51}
%% \infty
\DeclareSymbolFont{symbolsCM}{OMS}{cmsy}{m}{n}
\SetSymbolFont{symbolsCM}{bold}{OMS}{cmsy}{b}{n}
\let\txinfty\infty
\DeclareMathSymbol{\infty}{\mathord}{symbolsCM}{"31}
\definecolor{nuanbai}{HTML}{f5f5f5}
\pagecolor{lightgray!6}
% ---------------------------------------------------------------------------- %
% \overfullrule=5pt
% \showboxdepth=\maxdimen
% \showboxbreadth=\maxdimen
\tracingonline=1
\tracingoutput=1

\addbibresource{ref.bib}
% 定义紧密双内积符号
\newcommand{\llangle}{\left\langle\!\left\langle}
\newcommand{\rrangle}{\right\rangle\!\right\rangle}

% fullwidth environment, text across textwidth+marginparsep+marginparwidth
\newlength{\overhang}
\setlength{\overhang}{\marginparwidth}
\addtolength{\overhang}{\marginparsep}
%
\newenvironment{fullwidth}
  {\ifthenelse{\boolean{@twoside}}%
     {\begin{adjustwidth*}{}{-\overhang}}%
     {\begin{adjustwidth}{}{-\overhang}}%
  }%
  {\ifthenelse{\boolean{@twoside}}%
    {\end{adjustwidth*}}%
    {\end{adjustwidth}}%
  }
% ----------------------------------- 公式专用 ----------------------------------- %

\NewDocumentEnvironment{fulleq}{}%
  {
   \checkoddpage
   \ifoddpage
     % 奇数页:左侧调整宽度
     \begin{adjustwidth}{-\dimexpr\marginparwidth+\marginparsep\relax}{0pt}
   \else
     % 偶数页:右侧调整宽度
     \begin{adjustwidth}{0pt}{-\dimexpr\marginparwidth+\marginparsep\relax}
   \fi
   \noindent
  }%
  {\end{adjustwidth}}

% ---------------------------------------------------------------------------- %
\begin{document}
%----------------------------------------------------------------------------------------
%	标题页信息
%----------------------------------------------------------------------------------------
\title{Research Paper Comments}
\subtitle{\textit{\texorpdfstring{$L^q$}{L square}-extension theorem for jets on weakly pseudoconvex k\"ahler manifolds}}
\author{Ethan Lu}
\date{\today}
\publishers{Textbook}
%----------------------------------------------------------------------------------------

%----------------------------------------------------------------------------------------
%	插入自定义标题页
%----------------------------------------------------------------------------------------
\begin{titlepage} % 创建一个新的页面
%用来将图片从左下角开始平铺整个封面
	\AddToShipoutPicture*{%
	\AtPageLowerLeft{%
		\adjustbox{width=1.1\paperwidth, height=1.5\paperheight, keepaspectratio}{% 强制图片至纸张尺寸,但可能裁切
			\includegraphics{images/pexels-photo-3378916.jpeg}
		}
	}
}
\begin{flushleft} % 左对齐环境
	\setlength{\leftskip}{1cm} % 左侧缩进
	\makeatletter % 允许访问带有@字符的内部命令
	% \vspace*{4cm} % 标题距离页面顶端的空白
	% {\color{white}\Huge \textbf{\@title} \par} % 使用前文定义的title作为标题
	% \vspace{1cm} % 标题和子标题的间距
	% {\color{white}\Large \@subtitle \par} % 使用前文定义的subtitle作为子标题
	% \vfill % 作者信息前的垂直填充
	% {\color{white}\large \@author \par} % 作者名
	% {\color{white}\large \@publishers \par} % 出版者
	% {\color{white}\large \today\par} % 日期,默认为当天
        \begin{tikzpicture}[overlay,remember picture]
        \begin{pgfonlayer}{bottom}
            \fill[dblue!10,opacity=0.1] (current page.south west) rectangle ++(\paperwidth,2cm);
            \node[inner sep=0pt,text=white,font=\large\sffamily,above] (bottominfo) at ([yshift=.7cm]current page.south) {
                \@author\hspace{4cm}\@publishers\hspace{4cm}\today
            };
        \end{pgfonlayer}
        \fill[color=black!50,opacity=.2]node[append after command={
            ([yshift=0.5cm]bookinfo.north west) rectangle ([yshift=-0.5cm]bookinfo.south east)},minimum width=\paperwidth,opacity=1,align=left,inner sep=0pt,anchor=west] (bookinfo) at ([shift={(0,4cm)}]current page.west) {\hspace{-7cm}
                \begin{varwidth}{\linewidth}
                    \setlength\baselineskip{4ex}
                    \textcolor{black!10!white}{\Huge \textbf{\@title}} \\[.6cm]
                    \textcolor{black!10!white}{\Large \@subtitle}
                \end{varwidth}
                };
    \end{tikzpicture}
	\makeatother % 将@重置为非字母字符
	\vspace{0cm} % 标题和子标题的间距
	% 结束左对齐环境
\end{flushleft}
\ClearShipoutPicture % 清除背景图片
\end{titlepage}
% --------------------------------- 主要内容写在下面 --------------------------------- %
\pagestyle{Mainpage} % 页面样式
\chapimg{images/pexels-photo-1452701.jpeg}

\begin{titlepage}
    \newgeometry{left=2cm,right=2cm,top=2.5cm,bottom=2.2cm}
    \tableofcontents
    \restoregeometry
\end{titlepage}

% ---------------------------------------------------------------------------- %

\partimg{images/pexels-photo-931018.jpeg}
\part{Proof of Opennes Conjecture}


\chapter{\texorpdfstring{$L^q$}{L square}-extension theorem for jets on weakly pseudoconvex k\"ahler manifolds}

\section{Introduction}
T. Ohsawa-K. Takegoshi established a remarkable extension theorem of holomorphic functions defined on a bounded pseudoconvex domain in $\mathbb{C}^n$ with growth control in \cite{OhsawaTakegoshi1987}. Since then, many versions and variants of the $L^2$ extension theorems have been studied (see  \cite{BZ13,Manivel1993,MV06,Popovici2004L2EF,ZZ18}, etc.). These results lead to numerous applications in algebraic geometry and complex analysis.

One interesting problem is to study the $L^2$ extension theorem for jets. The first such result was given by D. Popovici \cite{Popovici2004L2EF}, which generalized the $L^2$ extension theorems of Ohsawa-Takegoshi-Manivel to the case of jets of sections of a line bundle over a weakly pseudoconvex Kähler manifold. Then J.-P. Demailly \cite{Demailly2015} considered the extension from more general non-reduced varieties. Following a new method of B. Berndtsson-L. Lempert \cite{BB14}, G. Hosono \cite{HG17} proved an $L^2$ extension theorem for jets with optimal estimate on a bounded pseudoconvex domain in $\mathbb{C}^n$ (see also \cite{MJV17}).

The idea of considering variable denominators was first introduced by J. McNeal-D. Varolin \cite{MV06}. They obtained some results on weighted $L^2$ extension of holomorphic top forms with values in a holomorphic line bundle, where the weights used are determined by the variable denominators. Recently, X. Zhou-L. Zhu \cite{ZZ18} proved an $L^2$ extension theorem for holomorphic sections of holomorphic line bundles equipped with singular metrics on weakly pseudoconvex Kähler manifolds. Furthermore, they obtained optimal constants corresponding to variable denominators.

The method of solving undetermined functions with ODEs was first used in \cite{GZ12}. From then on, a lot of spectacular works appear along this line, such as \cite{GZ13,ZZ18,GZ18}, etc. Several optimal $L^2$ extension theorems have been proved in this process.

The main goal of this paper\sidenote{An extended version of the theorem 1.2 in \cite{ZZ18} is demonstrated using the existing method.} is to apply the methods of Zhou-Zhu \cite{ZZ18} and Demailly \cite{Demailly2000} to $L^q$ jet extension to slightly generalize the theorem 1.2 in \cite{ZZ18}. As an application of our main theorem, we also obtain a corollary of a local $L^{\frac{2}{q}}$ extension theorem. 

We make precise the setting for our work. Let $X$ be an $n$-dimensional weakly pseudoconvex manifold with K\"ahler metric $\omega$, and $E$ a Hermitian holomorphic vector bundle of rank $m \geq 1$ over $X$. Assume that $s \in H^0(X, E)$ is transverse to the zero section. Set
$$
Y:=\{x \in X: s(x)=0\} .
$$

Furthermore, let $L$ be a holomorphic line bundle equipped with a smooth Hermitian metric satisfying an appropriate positivity condition.

We denote by $\bigwedge^{r, s} T_X^*$ the bundle of differential forms of bidegree $(r, s)$ on $X$, and $\mathcal{J}_Y$ the sheaf of germs of holomorphic functions on $X$ which vanish on $Y$. For any integer $k \geq 0$, let $\mathcal{O}_X / \mathcal{J}_Y^{k+1}$ be the nonlocally free sheaf of $k$-jets which are "transversal" to $Y$. Fix a point $y \in X$ and a Stein neighborhood $U$ in $X$ of $y$. Then this gives rise to a surjective morphism
$$
H^0\left(U, K_X \otimes L\right) \longrightarrow H^0\left(U, K_X \otimes L \otimes \mathcal{O}_X / \mathcal{J}_Y^{k+1}\right)
$$
of local section spaces, and an arbitrary local lifting $\tilde{f} \in H^0\left(U, K_X \otimes L\right)$ of $f$. For any transversal $k$-jet $f \in H^0\left(U, K_X \otimes L \otimes \mathcal{O}_X / \mathcal{J}_Y^{k+1}\right)$ and any weight function $\rho>0$ on $U$, the pointwise $\rho$-weighted norm associated to the section $s$, was defined by \cite[Definition 0.1.1]{Popovici2004L2EF}:
 \begin{fullwidth}
 $$
 |f|_{s, \rho,(k)}^2(y):=|\tilde{f}|_L^2(y)+\frac{\left|\nabla^1 \tilde{f}\right|_L^2}{\left|\bigwedge^m(\dd s)\right|_E^{2 \frac{1}{m}} \rho^{2(m+1)}}(y)+\cdots+\frac{\left|\nabla^k \tilde{f}\right|_L^2}{\left|\bigwedge^m(\dd s)\right|_E^{2 \frac{k}{m}} \rho^{2(m+k)}}(y),
 $$
\end{fullwidth}
and the $L_{(k)}^2$ weighted norm by:
$$
\|f\|_{s, \rho,(k)}^2=\int_Y \frac{|f|_{s, \rho,(k)}^2}{\left|\bigwedge^m(\dd s)\right|_E^2} \dd V_{Y, \omega} .
$$

Here for $i=0, \ldots, k, \nabla^i \widetilde{f}$ is constructed by induction as the projection of the $(1,0)$-part
$$
\nabla^{1,0}\left(\nabla^k \tilde{f}\right) \in C^{\infty}\left(U, K_X \otimes L \otimes S^{j-1} N_{Y / X}^* \otimes T_X^*\right)
$$
of $\nabla\left(\nabla^k \widetilde{f}\right)$ with the associated Chern connection $\nabla$ to $C^{\infty}\left(U, K_X \otimes L\right. $ $\left.\otimes S^j N_{Y / X}^*\right)$, induced by the surjective bundle morphism $\left.K_X \otimes L \otimes T_X^*\right|_Y$ $ \rightarrow K_X \otimes L \otimes N_{Y / X}^*$.

It is worthwhile to notice that the norm $|f|_{s, \rho,(k)}^2$ of the $k$-jet $f$ at the point $y \in Y$ is independent of the choice of the local lifting $\tilde{f}$. Moreover, one has the following notations \cite[Notation 0.1.3]{Popovici2004L2EF}:
\begin{enumerate}[label=(\alph*)]
  \item For a transversal $k$-jet $f \in H^0\left(U, K_X \otimes L \otimes \mathcal{O}_X / \mathcal{J}_Y^{k+1}\right)$, denote $\nabla^j f:=\left(\nabla^j \widetilde{f}\right)_{\mid U \cap Y}$, for all $j=0, \ldots, k$ and an arbitrary lifting $\tilde{f} \in H^0\left(U, K_X \otimes L\right)$ of $f$.
  \item For every integer $k \geq 0$, and every open set $U \subset X$, set
$$
J_U^k: H^0\left(U, K_X \otimes L\right) \longrightarrow H^0\left(U, K_X \otimes L \otimes \mathcal{O}_X / \mathcal{J}_Y^{k+1}\right)
$$
as the cohomology group morphism induced by the projection $\mathcal{O}_X \rightarrow \mathcal{O}_X / \mathcal{J}_Y^{k+1}$.
\end{enumerate}
We refer to \cite[pp. 2-5]{Popovici2004L2EF} for more details about the notations and the construction of relevant metrics on jets.

In \cite[p. 136]{ZZ18}, Zhou-Zhu defined the \emph{variable denominators}. Let $\frakR$ be the class of functions defined by
$$
\left\{R \in C^{\infty}(-\infty, 0]: \begin{array}{c}
R>0, \;R^{\prime} \leq 0,\; \int_{-\infty}^0 \frac{1}{R(t)} \mathrm{d} t<+\infty \\[.25em]
\text { and } e^t R(t) \text { is bounded above on }(-\infty, 0]
\end{array}\right\} .
$$

Denote $\int_{-\infty}^0 \frac{1}{R(t)} \mathrm{d} t$ by $C_R$. Notice that the function $R(t)$ equals to the function $\frac{1}{c_A(-t) e^t}$ defined just before the main theorems in \cite[p. 1143]{GZ12} when $A=0$.
With such preparation, our main theorem is as follows\sidenote{The main innovation of this paper is the extension of Theorem 1.2 from \cite{ZZ18} to the jets case.}.

%    Text of article.
\begin{theorem}[Main Theorem]\label{thm:main}
  Let $(X, \omega)$ be a weakly pseudoconvex complex n-dimensional manifold possessing a Kähler metric $\omega, \psi$ be a plurisubharmonic function on $X$, $E$ be a holomorphic vector bundle of rank $m$ over $X$ equipped with a smooth Hermitian metric $(1 \leq m\leq n)$, and $s$ be a global holomorphic section of $E$. Assume that $s$ is transverse to the zero section, and let
$$
Y:=\{x \in X: s(x)=0, \bigwedge^m (\dd s)(x)\neq 0\} .
$$

Let $L$ be a holomorphic line bundle over $X$ equipped with a singular Hermitian metric $h_L$, which is written locally as $e^{-\varphi_L}$ for some function $\varphi_L \in L_{\text {loc }}^1$ with respect to a local holomorphic frame of $L$. {\color{purple} Assume that $\frac{q}{2}\varphi_L+\br{1-\frac{q}{2}}\varphi_\omega+\psi$ is quasi-plurisubharmonic} and $\varphi_L$ is locally bounded above.  Let $0<q\leq 2$.  Moreover, assume that the $(1,1)$-form  \vspace{.25em}

{\color{purple} {\upshape (i)} \vspace{.25em}$\frac{q}{2}\sqrt{-1} \Theta_L+\br{1-\frac{q}{2}}\sqrt{-1}\bd\bdd\varphi_\omega+ m\sqrt{-1} \partial \bar{\partial} \log |s|^2+\xu\bd\bdd \psi $ $\geq 0$ 
holds on $X \backslash Y$,} 

and that there is a continuous function $\alpha>0$ on $X$ such that the following two inequalities hold on $X \backslash Y$ :

{\upshape (ii)} $\frac{q}{2}\displaystyle\sqrt{-1} \Theta_L+\br{1-\frac{q}{2}}\sqrt{-1}\bd\bdd\varphi_\omega+m\sqrt{-1} \partial \bar{\partial} \log |s|^2+\xu\bd\bdd \psi \geq \frac{\left\{\sqrt{-1} \Theta_E s, s\right\}_E}{\alpha|s|_E^2}$,

{\upshape (iii)} $\psi+m\log |s|^2\leq-2 m\alpha$ 

\noindent Then, for every relatively compact open subset $\Omega \subset X$, and every $k$-jet $f \in H^0\left(X, K_X\otimes L \otimes \mathcal{O}_X / \mathcal{J}_Y^{k+1}\right)$ satisfying
$$
C_f :=\int_Y \frac{|f|_{s, \rho,(k)}^q e^{-\psi}}{\left|\bigwedge^m(d s)\right|^{2}_E }d V_{Y, \omega}<+\infty.
$$
Furthermore, assume that there exists $F_1^{(k)} \in H^0\left(X, K_X \otimes L\right)$ such that $J^k F_1^{(k)} =f$ and
$$
C_{F_1}:=\int_{\Omega} \frac{\left|F_1^{(k)}\right|_L^q e^{-\psi}}{|s|^{2m} R(\psi+m\log |s|^2)} d V_{X, \omega} < +\infty.
$$
Then there exists $F^{(k)} \in H^0\left(X, K_X \otimes L\right)$ such that $J^k F^{(k)}=f$ and
$$
\int_{\Omega} \frac{\left|F^{(k)}\right|_L^q e^{-\psi}}{|s|^{2m} R(\psi+m\log |s|^2)} d V_{X, \omega} \leq C_{m,R}^{(k)} C_f,
$$
where $C_{m,R}^{(k)}>0$ is a constant depending only on $m,R,k, E$, and $\sup _{\Omega}\|i \Theta_L\|$.

\end{theorem}

Let $p$ be a positive integer. If we take $q=\frac{2}{p}$ and replace $L$ by $K_X^{p-1}\otimes L$, which is equipped with the metric $e^{(p-1)\varphi_\omega-\varphi_L}$, then we can get from Main Theorem the following corollary.

\begin{corollary}\label{cor:1}
  Assume that $\frac{\varphi_L}{p}+\psi$ is quasi-plurisubharmonic and $\varphi_L$ is locally bounded above. Moreover, assume that\vspace{.25em}

  {\upshape (i)}\vspace{.25em} \quad $\frac{\sqrt{-1} \Theta_L}{p}+\xu\bd\bdd\psi+m\sqrt{-1} \partial \bar{\partial} \log |s|^2 \geq 0$ holds on $X \backslash Y$,\newline
  and that there is a continuous function $\alpha>0$ on $X$ such that the following two inequalities hold on $X \backslash Y$ :

  {\upshape (ii)} \quad $\frac{\sqrt{-1} \Theta_L}{p}+\xu\bd\bdd\psi+m\sqrt{-1} \partial \bar{\partial} \log |s|^2\geq \frac{\left\{\sqrt{-1} \Theta_E s, s\right\}_E}{\alpha|s|_E^2}$,

  {\upshape (iii)} \quad $\psi+m\log |s|^2 \leq-2 m \alpha$.
  
For every relatively compact open subset $\Omega\subset X$, and every  $k$-jet  $f\in H^0 (X,K_X^p\otimes L\otimes \mO_X/\mJ_Y^{k+1})$, such that
  $$
  C_f:=\int_Y \frac{\left(|f|_L\right)^{\frac{2}{p}} e^{-\psi}}{\left|\bigwedge^m(d s)\right|_E^2} d V_Y<+\infty .
  $$
  Furthermore, assume that there exists a $k$-jet  $F_1^{(k)}\in H^0(\Omega,K_X^p\otimes L)$  such that $J^k F_1^{(k)}=f$ and
  $$
  C_{F_1}:=\int_\Omega \frac{\left(\left|F_1^{(k)}\right|_L\right)^{\frac{2}{p}} e^{-\psi}}{|s|^{2m} R(\psi+m\log |s|^2)} d V_X<+\infty.
  $$
  Then there exists a $k$-jet  $F^{(k)}\in H^0(\Omega,K_X^p\otimes L)$, such that $J^k F^{(k)}=f$ and
$$
\int_\Omega \frac{\left(|F^{(k)}|_L\right)^{\frac{2}{p}} e^{-\psi}}{|s|^{2m} R(\psi+m\log |s|^2)} d V_X \leq C_{m,R}^{(k)}  C_f ,
$$
where $C_{m,R}^{(k)}>0$ is a constant depending only on $m,R,k, E$, and $\sup _{\Omega}\|i \Theta_L\|$.
\end{corollary}


\section{Preliminaries}
\begin{lemma}[Baisc a priori inequality]\label{lem:priori-estimate}
  Let $E$ be a hermitian vector bundle on a complex manifold $X$ equipped with a Kähler metric $\omega$. Let $\eta, \lambda>0$ be smooth functions on $X$. Then for every form $u \in \mathcal{D}\left(X, \bigwedge^{p, q} T_X^{\star} \otimes E\right)$ with compact support we have
$$
\begin{aligned}
&\left\|\left(\eta^{\frac{1}{2}}+\lambda^{\frac{1}{2}}\right) D^{\prime \prime *} u\right\|^2  +\left\|\eta^{\frac{1}{2}} D^{\prime \prime} u\right\|^2+\left\|\lambda^{\frac{1}{2}} D^{\prime} u\right\|^2+2\left\|\lambda^{-\frac{1}{2}} d^{\prime} \eta \wedge u\right\|^2 \\
& \geqslant\llangle\left[\eta \mathrm{i} \Theta(E)-\mathrm{i} d^{\prime} d^{\prime \prime} \eta-\mathrm{i} \lambda^{-1} d^{\prime} \eta \wedge d^{\prime \prime} \eta, \Lambda\right] u, u\rrangle .
\end{aligned}
$$
\end{lemma}


\begin{lemma}[$L^2$-existence theorem with erro term]\label{lem:L2existence}
  Let $(X, \omega)$ be a complete K\"ahler manifold equipped with a (non-necessarily complete) K\"ahler metric $\omega$, and let $Q$ be a Hermitian vector bundle over $X$. Assume that $\tau$ and $A$ are smooth and bounded positive functions on $X$ and let $$\mathrm{B}:=\left[\tau \sqrt{-1} \Theta_Q-\sqrt{-1} \partial \bar{\partial} \tau-\sqrt{-1} A^{-1} \partial \tau \wedge \bar{\partial} \tau, \Lambda\right].$$ Assume that $\delta \geq 0$ is a number such that $\mathrm{B}+\delta \mathrm{I}$ is semi-positive definite everywhere on $\bigwedge^{n, q} T_X^* \otimes Q$ for some $q \geq 1$. Then given a form $g \in L^2\left(X, \bigwedge^{n, q} T_X^* \otimes Q\right)$ such that $\mathrm{D}^{\prime \prime} g=0$ and $$\int_X\left\langle(\mathrm{~B}+\delta \mathrm{I})^{-1} g, g\right\rangle_Q d V_X<+\infty,$$ there exists an approximate solution $u \in L^2\left(X, \bigwedge^{n, q-1} T_X^* \otimes Q\right)$ and a correcting term $h \in L^2\left(X, \right.$ $\left.\bigwedge^{n, q} T_X^* \otimes Q\right)$ such that $\mathrm{D}^{\prime \prime} u+\sqrt{\delta} h=g$ and
$$
\int_X \frac{|u|_Q^2}{\tau+A} d V_X+\int_X|h|_Q^2 d V_X \leq \int_X\left\langle(\mathrm{~B}+\delta \mathrm{I})^{-1} g, g\right\rangle_Q d V_X .
$$
\end{lemma}

\begin{proof}
  By lemma \ref{lem:priori-estimate}, lemma \ref{lem:L2existence} can be obtained by almost the same arguments as in \cite{Demailly2000}, where the term $\int_X\left\langle(\mathrm{~B}+\delta \mathrm{I})^{-1} g, g\right\rangle_Q d V_X$ in the above inequality is written as $2 \int_X\left\langle(\mathrm{~B}+\delta \mathrm{I})^{-1} g, g\right\rangle_Q d V_X$.
\end{proof}

% \begin{proof}[The proof in \cite{Demailly2000}]
%   Let $v \in L^2\left(X, \Lambda^{n, q} T_X^{\star} \otimes E\right)$ be an arbitrary element. Assume first that $\omega$ is complete, so that $\left(\operatorname{Ker} D^{\prime \prime}\right)^{\perp}=\overline{\operatorname{Im} D^{\prime \prime *}} \subset \operatorname{Ker} D^{\prime \prime *}$. Then, by using the decomposition $v=v_1+v_2 \in\left(\operatorname{Ker} D^{\prime \prime}\right) \oplus\left(\operatorname{Ker} D^{\prime \prime}\right)^{\perp}$ and the fact that $g \in \operatorname{Ker} D^{\prime \prime}$, we infer from Cauchy-Schwarz the inequality
% $$
% |\langle g, v\rangle|^2=\left|\left\langle g, v_1\right\rangle\right|^2 \leqslant \int_X\left\langle B^{-1} g, g\right\rangle d V_\omega \int_X\left\langle B v_1, v_1\right\rangle d V_\omega .
% $$

% We have $v_2 \in \operatorname{Ker} D^{\prime \prime *}$, hence $D^{\prime \prime *} v=D^{\prime \prime *} v_1$, and \autoref{lem:priori-estimate} implies
% $$
% \int_X\left\langle B v_1, v_1\right\rangle d V_\omega \leqslant\left\|\left(\eta^{\frac{1}{2}}+\lambda^{\frac{1}{2}}\right) D^{\prime \prime *} v_1\right\|^2+\left\|\eta^{\frac{1}{2}} D^{\prime \prime} v_1\right\|^2=\left\|\left(\eta^{\frac{1}{2}}+\lambda^{\frac{1}{2}}\right) D^{\prime \prime \star} v\right\|^2
% $$
% provided that $v \in \operatorname{Dom} D^{\prime \prime \star}$. Combining both, we find
% $$
% |\langle g, v\rangle|^2 \leqslant\left(\int_X\left\langle B^{-1} g, g\right\rangle d V_\omega\right)\left\|\left(\eta^{\frac{1}{2}}+\lambda^{\frac{1}{2}}\right) D^{\prime \prime *} v\right\|^2 .
% $$

% This shows the existence of an element $w \in L^2\left(X, \Lambda^{n, q} T_X^{\star} \otimes E\right)$ such that
% $$
% \begin{aligned}
% \|w\|^2 & \leqslant \int_X\left\langle B^{-1} g, g\right\rangle d V_\omega \quad \text { and } \\
% \langle\langle v, g\rangle\rangle & =\left\langle\left(\eta^{\frac{1}{2}}+\lambda^{\frac{1}{2}}\right) D^{\prime \prime \star} v, w\right\rangle \quad \forall g \in \operatorname{Dom} D^{\prime \prime} \cap \operatorname{Dom} D^{\prime \prime *} .
% \end{aligned}
% $$

% As $\left(\eta^{1 / 2}+\lambda^{\frac{1}{2}}\right)^2 \leqslant 2(\eta+\lambda)$, it follows that $f=\left(\eta^{1 / 2}+\lambda^{\frac{1}{2}}\right) w$ satisfies $D^{\prime \prime} f=g$ as well as the desired $L^2$ estimate. If $\omega$ is not complete, we set $\omega_{\varepsilon}=\omega+\varepsilon \widehat{\omega}$ with some complete Kähler metric $\widehat{\omega}$. The final conclusion is then obtained by passing to the limit and using a monotonicity argument (the integrals are monotonic with respect to $\varepsilon$ ). The technique is quite standard and entirely similar to the approach described in [Dem82a], so we will not give any detail here.
% \end{proof}


\begin{lemma}[The property of psh function]
  Let $X$ be a Stein manifold and $\varphi$ be a plurisubharmonic function on $X$. Then there exists a decreasing sequence of smooth strictly plurisubharmonic functions $\left\{\varphi_j\right\}_{j=1}^{+\infty}$ such that $\lim _{j \rightarrow+\infty} \varphi_j=\varphi$
\end{lemma}

\begin{lemma}[Theorem 1.5 in \cite{Demailly1982}]\label{lem:3.5}
  Let $X$ be a Kähler manifold, and $Z$ be an analytic subset of $X$. Assume that $\Omega$ is a relatively compact open subset of $X$ possessing a complete Kähler metric. Then $\Omega \backslash Z$ carries a complete Kähler metric.
\end{lemma}

\begin{lemma}[Theorem 4.4.2 in \cite{Hormander1990}]
  Let $\Omega$ be a pseudoconvex open set in $\mathbb{C}^n$, and $\varphi$ be a plurisubharmonic function on $\Omega$. For every $h \in$ $L_{(p, q+1)}^2(\Omega, \varphi)$ with $\bar{\partial} h=0$ there is a solution $v \in L_{(p, q)}^2(\Omega, \textrm{loc} )$ of the equation $\bar{\partial} v=h$ such that
$$
\int_{\Omega} \frac{|v|^2}{\left(1+|z|^2\right)^2} e^{-\varphi} d V \leq \int_{\Omega}|h|^2 e^{-\varphi} d V
$$
\end{lemma}

\begin{lemma}[Lemma 6.9 in \cite{Demailly1982}]\label{lem:6}
  Let $\Omega$ be an open subset of $\mathbb{C}^n$ and $Z$ be a complex analytic subset of $\Omega$. Assume that $v$ is a $(p, q-1)$-form with $L_{\text {loc }}^2$ coefficients and $h$ is a $(p, q)$-form with $L_{\textrm{loc}}^1$ coefficients such that $\bar{\partial} v=h$ on $\Omega \backslash Z$ (in the sense of distribution theory). Then $\bar{\partial} v=h$ on $\Omega$.
\end{lemma}

% \begin{lemma}[strong openness conjecture, see [17]]
%   Let $\varphi$ be a negative plurisubharmonic function on the unit polydisk $\Delta^n \subset \mathbb{C}^n$. Assume that $F$ is a holomorphic function on $\Delta^n$ satisfying
%   $$
%   \int_{\Delta^n}|F|^2 e^{-\varphi} d V_n<+\infty,
%   $$
%   where $d V_n$ is the $2 n$-dimensional Lebesgue measure on $\mathbb{C}^n$. Then there exists $r \in(0,1)$ and $\beta \in(0,+\infty)$ such that
%   $$
%   \int_{\Delta_r^n}|F|^2 e^{-(1+\beta) \varphi} d V_n<+\infty,
%   $$
%   where $\Delta_r^n:=\left\{\left(z^1, \cdots, z^n\right) \in \mathbb{C}^n:\left|z^k\right|<r, 1 \leq k \leq n\right\}$.
% \end{lemma}

\begin{lemma}[Lagrange's inequality]\label{lem:lagrange}
  Let $X$ be a complex manifold, $E$ be a Hermitian vector bundle over $X$ of rank $m$, and $\{\bullet, \bullet\}_E: \wedge^{p_1, q_1} T_X^* \otimes$ $E \times \wedge^{p_2, q_2} T_X^* \otimes E \longrightarrow \wedge^{p_1+q_2, q_1+p_2} T_X^*$ be the sesquilinear product which combines the wedge product $(u, v) \mapsto u \wedge \bar{v}$ on scalar valued forms with the Hermitian inner product on $E$. Then for any smooth section $s$ of $E$ over $X$ and any smooth section $w$ of $T_X^* \otimes E$ over $X$,
\begin{equation}\label{eq:3.1}
    \sqrt{-1}\{w, s\}_E \wedge\{s, w\}_E \leq|s|_E^2 \sqrt{-1}\{w, w\}_E .
\end{equation}
\end{lemma}

\begin{proof}
  Since $\{\bullet, \bullet\}_E$ is a pointwise product, it's sufficient to prove \eqref{eq:3.1}  at every fixed point of $X$. Hence, we can regard $T_X^*$ and $E$ as vector spaces. Then $s$ and $w$ are regarded as elements in $E$ and $T_X^* \otimes E$ respectively. If $s=0$, \eqref{eq:3.1} is trivial. If $s \neq 0$, without loss of generality, we can assume that $|s|_E=1$. Then we choose $e_2, \cdots, e_m \in E$ such that $s, e_2, \cdots, e_m$ form an orthonormal basis of $E$. Now $w$ can be written as
  $$
  w_1 \otimes s+\sum_{j=2}^m w_j \otimes e_j,
  $$
  for some $w_j \in T_X^*(1 \leq j \leq m)$. Then we have
  $$
  \sqrt{-1}\{w, s\}_E \wedge\{s, w\}_E=\sqrt{-1} w_1 \wedge \bar{w}_1,
  $$
  and
  $$
  |s|_E^2 \sqrt{-1}\{w, w\}_E=\sqrt{-1} \sum_{j=1}^m w_j \wedge \bar{w}_j \geq \sqrt{-1} w_1 \wedge \bar{w}_1 .
  $$
  
  Hence, \eqref{eq:3.1} holds. The lemma is, thus, proved.
\end{proof}


\section{Proof of the normal case theorem}
In order to prove the main theorem, we should prove the following theorem firstly.
\begin{theorem}[The case of $L^2$-extension]\label{thm:L2}
  Let $(X, \omega)$ be a weakly pseudoconvex complex $n$ dimensional manifold possessing a Kähler metric $\omega, \psi$ be a plurisubharmonic function on $X$, $E$ be a holomorphic vector bundle of rank $m$ over $X$ equipped with a smooth Hermitian metric $(1 \leq m\leq n)$, and $s$ be a global holomorphic section of $E$. Assume that $s$ is transverse to the zero section, and let
$$
Y:=\{x \in X: s(x)=0, \bigwedge^m (\dd s)(x)\neq 0\} .
$$

Let $L$ be a holomorphic line bundle over $X$ equipped with a singular Hermitian metric $h_L$, which is written locally as $e^{-\varphi_L}$ for some function $\varphi_L \in L_{\text {loc }}^1$ with respect to a local holomorphic frame of $L$. Assume that $\frac{q}{2}\varphi_L+\br{1-\frac{q}{2}}\varphi_\omega+\psi$ is quasi-plurisubharmonic and $\varphi_L$ is locally bounded above.  Moreover, assume that the $(1,1)$-form  \vspace{.25em}

{\upshape (i)}\vspace{.25em} $\frac{q}{2}\sqrt{-1} \Theta_L+\br{1-\frac{q}{2}}\sqrt{-1}\bd\bdd\varphi_\omega+ m\sqrt{-1} \partial \bar{\partial} \log |s|^2+\xu\bd\bdd \psi $\\$\geq 0$ holds on $X \backslash Y$, \newline
and that there is a continuous function $\alpha>0$ on $X$ such that the following two inequalities hold on $X \backslash Y$ :

{\upshape (ii)} $\frac{q}{2}\displaystyle\sqrt{-1} \Theta_L+\br{1-\frac{q}{2}}\sqrt{-1}\bd\bdd\varphi_\omega+m\sqrt{-1} \partial \bar{\partial} \log |s|^2+\xu\bd\bdd \psi \geq \frac{\left\{\sqrt{-1} \Theta_E s, s\right\}_E}{\alpha|s|_E^2}$,

{\upshape (iii) }$\psi+m\log |s|^2\leq-2 m\alpha$ 

\noindent Then, for every relatively compact open subset $\Omega \subset X$, and every $k$-jet $f \in H^0\left(X, K_X\otimes L \otimes \mathcal{O}_X / \mathcal{J}_Y^{k+1}\right)$ satisfying
$$
C_f :=\int_Y \frac{|f|_{s, \rho,(k)}^2 e^{-\psi} }{\left|\bigwedge^m(d s)\right|^{2}_E }d V_{Y, \omega}<+\infty.
$$

Then there exists $F^{(k)} \in H^0\left(X, K_X \otimes L\right)$ such that $J^k F^{(k)}=f$ and
$$
\int_{\Omega} \frac{\left|F^{(k)}\right|_L^2}{e^{\psi+m\log |s|^2_E} R(\psi+m\log |s|^2_E)} d V_{X, \omega} \leq C_{m,R}^{(k)} C_f,
$$
where $C_{m,R}^{(k)}>0$ is a constant depending only on $m,R,k, E$, and $\sup _{\Omega}\|i \Theta_L\|$.

\end{theorem}
\begin{proof}
Without loss of generality, we can suppose that $C_R=1$. Otherwise, we replace $R$ with $C_R R$ in the proof.

If $f=0$ on $Y$, then $F=0$ satisfies the conclusion of Proposition 4.1. In the following proof, we assume that $f$ is not 0 identically.

Since $X$ is pseudoconvex, there exists a smooth strictly plurisubharmonic exhaustion function $P$ on $X$. Instead of working on $X$ itself, we will work rather on the relatively compact subset $X_c \backslash Y$, where $X_c=\{P<c\}$ $\left(c=1,2, \cdots\right.$, we choose $P$ such that $\left.X_1 \neq \emptyset\right)$. By Lemma \ref{lem:3.5} $X_c \backslash Y$ $(c=1,2, \cdots)$ are complete Kähler.

We will discuss for fixed $c$ until the end of the proof.

% Since $X$ is Stein, by Cartan's Theorem B, there exists a holomorphic section $\tilde{f}$ on $X$ with values in $K_X \otimes L$ such that $\tilde{f}=f$ on $Y$.

Let $\zeta:(-\infty, 0) \longrightarrow(0,+\infty)$ be a smooth strictly increasing function, and $\chi:(-\infty, 0) \longrightarrow(0,+\infty)$ a smooth strictly decreasing function. Assume that $\chi(t) \geq-\frac{t}{2}$ for $t \in(-\infty, 0)$. We will find more assumptions about $\zeta$ and $\chi$ in the proof, by which we will get explicit $\zeta$ and $\chi$ in the end of this section.

Let $a \in(0,1)$ and put $\sigma_{\varepsilon}=m \log \left(|s|^2+\varepsilon^2\right)-a$ and $\sigma=m \log |s|^2-a$. Since $|s| \leq 1$ on $X$, there exists a positive number $\varepsilon_a \in(0,1)$ such that $\sigma_{\varepsilon} \leq-\frac{a}{2}$ on $\overline{X_c}$ for $\varepsilon \in\left(0, \varepsilon_a\right)$.

% By Lemma 3.3, there exists a decreasing sequence of smooth plurisubharmonic functions $\left\{\varphi_j\right\}_{j=1}^{+\infty}$ on $X$ such that $\lim _{j \rightarrow+\infty} \varphi_j=\varphi$. 
Assume that $K_X$ is naturally equipped with the smooth metric $e^{\varphi_\omega}$ with respect to the dual frame of $\dd z$. 
% Let $L$ be the line bundle $L$ equipped with the new metric $e^{-\varphi_{L}}$, where $\varphi_{L}:=\frac{q}{2}\varphi_L+\left(1-\frac{q}{2}\right)\varphi_\omega+\psi$.  
% Then the assumptions in the theorem imply that 
% \begin{enumerate}[label=(\roman*)]
%     \item $\xu \Theta_{L}+m\sqrt{-1} \partial \bar{\partial} \log |s|^2+\xu\bd\bdd \psi\geqslant 0$,
%     \item $\xu \Theta_{L}+m\sqrt{-1} \partial \bar{\partial} \log |s|^2+\xu\bd\bdd \psi\geqslant \frac{\{\xu\Theta_{E}s,s\}_E}{\alpha\abs{s}^2_E}.$  
% \end{enumerate}     
Let $L_{a, \varepsilon}$  denote the line bundle $L$ on $X_c \backslash Y$ equipped with the new metric $h_{a, \varepsilon}:=e^{-\br{\frac{q}{2}\varphi_L+\left(1-\frac{q}{2}\right)\varphi_\omega+\psi}-\sigma-\zeta\left(\sigma_{\varepsilon}\right)}$.

Set $\tau_{\varepsilon}=\chi\left(\sigma_{\varepsilon}\right)$ and let $A_{\varepsilon}$ be a smooth positive function on $\overline{X_c}$, which will be determined later. Set $\mathrm{B}_{\varepsilon}=\left[\Theta_{\varepsilon}, \Lambda\right]$ on $X_c \backslash Y$, where
$$
\Theta_{\varepsilon}:=\tau_{\varepsilon} \sqrt{-1} \Theta_{L_{a, \varepsilon}}-\sqrt{-1} \partial \bar{\partial} \tau_{\varepsilon}-\sqrt{-1} \frac{\partial \tau_{\varepsilon} \wedge \bar{\partial} \tau_{\varepsilon}}{A_{\varepsilon}} .
$$

Setting 
\begin{equation}
  \label{eq:3}
  \nu_\varepsilon:=\frac{\{D's,s\}}{|s|^2+\varepsilon^2}.
\end{equation}
{\color{purple} We want to find suitable $\zeta, \chi$ and $A_{\varepsilon}$ such that
\begin{equation}
  \label{eq:expected}
\left.\Theta_{\varepsilon}\right|_{X_c \backslash Y} \geq \frac{m \varepsilon^2}{|s|^2} \sqrt{-1} \nu_{\varepsilon} \wedge \bar{\nu}_{\varepsilon} .
\end{equation}}

Simple calculations yield
 \begin{fullwidth}
 \begin{equation}\label{eq:estimate}
 \begin{aligned}
 & \left.\Theta_{\varepsilon}\right|_{X_c \backslash Y} \\
 = & \chi\left(\sigma_{\varepsilon}\right)\left(\frac{q}{2}\sqrt{-1} \partial \bar{\partial} \varphi_{L}+\left(1-\frac{q}{2}\right)\xu\bd\bdd\varphi_\omega+\xu\bd\bdd\psi+\sqrt{-1} \partial \bar{\partial} \sigma\right) \\
 &+\left(\chi\left(\sigma_{\varepsilon}\right) \zeta^{\prime}\left(\sigma_{\varepsilon}\right)-\chi^{\prime}\left(\sigma_{\varepsilon}\right)\right) \sqrt{-1} \partial \bar{\partial} \sigma_{\varepsilon} +\left(\chi\left(\sigma_{\varepsilon}\right) \zeta^{\prime \prime}\left(\sigma_{\varepsilon}\right)-\chi^{\prime \prime}\left(\sigma_{\varepsilon}\right)-\frac{\left(\chi^{\prime}\left(\sigma_{\varepsilon}\right)\right)^2}{A_{\varepsilon}}\right) \sqrt{-1} \partial \sigma_{\varepsilon} \wedge \bar{\partial} \sigma_{\varepsilon} \\
 = & \chi\br{\sigma_\varepsilon}\left(\frac{q}{2}\sqrt{-1} \partial \bar{\partial} \varphi_{L}+\left(1-\frac{q}{2}\right)\xu\bd\bdd\varphi_\omega+\xu\bd\bdd\psi+m\sqrt{-1} \partial \bar{\partial} \log |s|^2\right)\\
 & +\left(\chi\left(\sigma_{\varepsilon}\right) \zeta^{\prime}\left(\sigma_{\varepsilon}\right)-\chi^{\prime}\left(\sigma_{\varepsilon}\right)\right) \sqrt{-1} \partial \bar{\partial} \sigma_{\varepsilon} +\left(\chi\left(\sigma_{\varepsilon}\right) \zeta^{\prime \prime}\left(\sigma_{\varepsilon}\right)-\chi^{\prime \prime}\left(\sigma_{\varepsilon}\right)-\frac{\left(\chi^{\prime}\left(\sigma_{\varepsilon}\right)\right)^2}{A_{\varepsilon}}\right) \sqrt{-1} \partial \sigma_{\varepsilon} \wedge \bar{\partial} \sigma_{\varepsilon} .
 \end{aligned}
 \end{equation}
\end{fullwidth}

{\color{dblue} Assume that the equalities
\begin{equation}\label{eq:5}
  \chi\left(\sigma_{\varepsilon}\right) \zeta^{\prime}\left(\sigma_{\varepsilon}\right)-\chi^{\prime}\left(\sigma_{\varepsilon}\right)=1
\end{equation}
and
\begin{equation}\label{eq:6}
  \chi\left(\sigma_{\varepsilon}\right) \zeta^{\prime \prime}\left(\sigma_{\varepsilon}\right)-\chi^{\prime \prime}\left(\sigma_{\varepsilon}\right)-\frac{\left(\chi^{\prime}\left(\sigma_{\varepsilon}\right)\right)^2}{A_{\varepsilon}}=0
\end{equation}}
hold, we obtain that
\begin{equation}\label{eq:1.3}
  \left.\Theta_{\varepsilon}\right|_{X_c \backslash Y} \geq \chi\br{\sigma_\varepsilon}\left(\frac{q}{2}\sqrt{-1} \partial \bar{\partial} \varphi_{L}+\left(1-\frac{q}{2}\right)\xu\bd\bdd\varphi_\omega+\xu\bd\bdd\psi+m\sqrt{-1} \partial \bar{\partial} \log |s|^2\right)+{\color{purple} \sqrt{-1} \partial \bar{\partial} \sigma_{\varepsilon} }.
\end{equation}\sidenote{We will provide the estimate for the last term.}

Furthermore, by \eqref{eq:5} we can {\color{purple} assume that $A_{\varepsilon}=\eta\left(\sigma_{\varepsilon}\right)$} for some smooth function $\eta:(-\infty, 0) \longrightarrow(0,+\infty)$ such that\sidenote{Here \eqref{eq:8} is equivalent to the above \eqref{eq:6}.}
{\color{dblue} \begin{equation}\label{eq:8}
  \chi \zeta^{\prime \prime}-\chi^{\prime \prime}-\frac{\left(\chi^{\prime}\right)^2}{\eta}=0
\end{equation}}

Since it follows from Lemma \ref{lem:lagrange} that
$$
|s|^2 \sqrt{-1} \sum_{i=1}^m d s^i \wedge d \bar{s}^i \geq \sqrt{-1}\left(\sum_{i=1}^m \bar{s}^i d s^i\right) \wedge\left(\sum_{i=1}^m s^i d \bar{s}^i\right)
$$
, which can be stated as
\[
  |s|^2 \sqrt{-1} \{D's ,D's\} \geq \sqrt{-1} \{D's,s\}\wedge \{s,D's\}.
\]
We obtain that on $X_c\backslash Y$,
\begin{align*}
 {\color{purple}  \xu\bd\bdd\sigma_\varepsilon} &= \frac{m\xu \{D's ,D's\} }{\abs{s}^2+\varepsilon^2}-\frac{m \sqrt{-1} \{D's,s\}\wedge \{s,D's\} }{(\abs{s}^2+\varepsilon^2)^2}-\frac{m\xu \{\Theta_E s,s\}}{\abs{s}^2+\varepsilon^2}\\ 
  &\geqslant \frac{m \varepsilon^2}{|s|^2}\frac{\sqrt{-1} \{D's,s\}\wedge \{s,D's\} }{(\abs{s}^2+\varepsilon^2)^2}-\frac{m\xu \{\Theta_E s,s\}}{\abs{s}^2+\varepsilon^2}\\ 
  &=\frac{m \varepsilon^2}{|s|^2}\xu \nu_\varepsilon\wedge \bar\nu_{\varepsilon}-\frac{m\xu \{\Theta_E s,s\}}{\abs{s}^2+\varepsilon^2}.
\end{align*} 

Then it follows from \eqref{eq:1.3} that on $X_c\backslash Y$, 
\[
\begin{aligned}
    \Theta_\varepsilon \geqslant &\br{\chi(\sigma_\varepsilon) \left(\frac{q}{2}\sqrt{-1} \partial \bar{\partial} \varphi_{L}+\left(1-\frac{q}{2}\right)\xu\bd\bdd\varphi_\omega+\xu\bd\bdd\psi+m\sqrt{-1} \partial \bar{\partial} \log |s|^2\right)
    -\frac{m\xu \{\Theta_E s,s\}}{\abs{s}^2+\varepsilon^2}}\\
    +&\frac{m \varepsilon^2}{|s|^2}\xu \nu_\varepsilon\wedge \bar\nu_{\varepsilon}.
\end{aligned}
\] 

Since $\chi (\sigma_\varepsilon)\geq m\alpha$ by the assumption $\chi(t)\geq -\frac{t}{2}$  , it follows from the condition on $X\backslash Y$ in Theorem \ref{thm:main}  that 
\begin{align*}
  &\chi (\sigma_\varepsilon) (\frac{q}{2}\sqrt{-1} \partial \bar{\partial} \varphi_{L}+\left(1-\frac{q}{2}\right)\xu\bd\bdd\varphi_\omega+\xu\bd\bdd\psi+m\xu\bd\bdd \log |s|^2)-\frac{m\xu \{\Theta_E s,s\}}{\abs{s}^2+\varepsilon^2}\\ 
& =\chi (\sigma_\varepsilon) (\frac{q}{2}\sqrt{-1} \partial \bar{\partial} \varphi_{L}+\left(1-\frac{q}{2}\right)\xu\bd\bdd\varphi_\omega+\xu\bd\bdd\psi+m\xu\bd\bdd \log |s|^2)-\frac{m\alpha\abs{s}^2 }{\abs{s}^2+\varepsilon^2}\frac{\xu \{\Theta_E s,s\}}{\alpha \abs{s}^2}\\ 
&\geq \frac{m\alpha\abs{s}^2 }{\abs{s}^2+\varepsilon^2} \br{\frac{q}{2}\sqrt{-1} \partial \bar{\partial} \varphi_{L}+\left(1-\frac{q}{2}\right)\xu\bd\bdd\varphi_\omega+\xu\bd\bdd\psi+\xu\bd\bdd \sigma-\frac{\xu \{\Theta_E s,s\}}{\alpha \abs{s}^2}}\\ 
&\geq 0 \quad \text{(By the assumption (ii) in Theorem \ref{thm:main}.)}
\end{align*}
on $X_c\backslash Y$ . Hence, one obtain \eqref{eq:expected} as expected.

{\color{purple} As a result, we have 
\begin{equation}\label{eq:9}
  B_\varepsilon\geq \left[\frac{m\varepsilon^2}{\abs{s}^2}\xu\nu_\varepsilon\wedge\bar\nu_\varepsilon,\Lambda\right]=\frac{m\varepsilon^2}{\abs{s}^2}T_{\bar\nu_\varepsilon}T_{\bar\nu_\varepsilon}^*
\end{equation}}
on $X_c\backslash Y$ as an operator on $(n,1)$ forms, where $T_{\bar\nu_\varepsilon}$  denotes the operator $\bar\nu_\varepsilon\wedge\bullet$ and $T_{\bar\nu_\varepsilon}^*$ is its Hilbert adjoint operator.

\subsection{Solving \texorpdfstring{$\bar{\partial}$}{}-equation on \texorpdfstring{$X_c$}{} with estimates.}

With such preparation, we now argue by {\color{purple} induction on $k \geq 0$}. The case $k=0$ is a special case of \cite[Theorem 1.2]{ZZ18}. Now, {\color{purple} assume that the theorem has been proved for $k-1$}, and we consider the short exact sequence of sheaves
$$
0 \longrightarrow S^k N_{Y / X}^* \longrightarrow \mathcal{O}_X / \mathcal{J}_Y^{k+1} \longrightarrow \mathcal{O}_X / \mathcal{J}_Y^k \longrightarrow 0
$$

Let $J^{k-1} f \in H^0\left(X, K_X \otimes L \otimes \mathcal{O}_X / \mathcal{J}_Y^k\right)$ be the image of $f \in H^0\left(X, K_X \otimes L \otimes \mathcal{O}_X / \mathcal{J}_Y^{k+1}\right)$ under the induced cohomology group morphism. By {\color{purple} the induction hypothesis}\sidenote{By theorem \ref{thm:L2}, there exists $F^{(k-1)} \in H^0\left(X, K_X \otimes L\right)$ such that $J^{k-1} F^{(k-1)}=J^{k-1}f$ and
{\footnotesize $$
 \begin{aligned}
 &\int_{\Omega} \frac{\left|F^{(k-1)}\right|_L^2}{e^{\psi+m\log |s|^2_E} R(\psi+m\log |s|^2_E)} d V_{X, \omega} \\
 &\leq C_{m,R}^{(k-1)} C_f,
\end{aligned}
$$}
where $C_{m,R}^{(k-1)}>0$ is a constant depending only on $m,R,k, E$, and $\sup _{\Omega}\|i \Theta_L\|$.}, {\color{dblue} there exists $F^{(k-1)} \in$ $H^0\left(X, \right.$ $\left. K_X \otimes L\right)$ such that
{\small\begin{equation}\label{eq:10}
 \begin{gathered}
    J^{k-1} F^{(k-1)}=J^{k-1} f, \\\int_{X_c} \frac{\left|F^{(k-1)}\right|_L^2}{|s|^{2 m} R\left(\psi+m \log |s|_E^2\right)} e^{-\psi}\mathrm{d} V_{X, \omega} \leq C_{m, R}^{(k-1)} \int_{Y_c} \frac{|f|_{s, \rho,(k-1)}^2 e^{-\psi}}{\left|\Lambda^m(\mathrm{~d} s)\right|^2} \mathrm{~d} V_{Y, \omega},
\end{gathered}
\end{equation}}
where $C_{m, R}^{(k-1)}>0$ is a constant as in the statement of Theorem \ref{thm:L2} and $Y_c:=Y \cap X_c$.} Thus, the image $$J^{k-1} f-J^{k-1} F^{(k-1)} \in H^0\left(X, K_X \otimes L \otimes \mathcal{O}_X / \mathcal{J}_Y^k\right)$$ of $f-J^k F^{(k-1)} \in H^0\left(X, K_X \otimes L \otimes \mathcal{O}_X / \mathcal{J}_Y^{k+1}\right)$ vanishes. So we can view the jet $f-J^k F^{(k-1)}$ as a global holomorphic section (on $Y$ ) of the sheaf $K_X \otimes L \otimes S^k N_{Y / X}^*=K_X \otimes L \otimes S^k E_{\mid Y}^*$\sidenote{The explanation:

 \textbf{1. Short Exact Sequence and Structure:}
The given short exact sequence  
\[
 \begin{aligned}
 &0 \longrightarrow S^k N_{Y/X}^* \longrightarrow \mathcal{O}_X / \mathcal{J}_Y^{k+1} \\
 &\longrightarrow \mathcal{O}_X / \mathcal{J}_Y^k \longrightarrow 0
\end{aligned}
\]
shows that the difference between the sheaves \( \mathcal{O}_X / \mathcal{J}_Y^{k+1} \) and \( \mathcal{O}_X / \mathcal{J}_Y^k \) lies in sections of \( S^k N_{Y/X}^* \). 

 \textbf{2. Constructing the Section:}
Given \( f \in H^0(X, K_X \otimes L \otimes \mathcal{O}_X / \mathcal{J}_Y^{k+1}) \), its image under the induced map is \( J^{k-1} f \) in \( H^0(X, K_X \otimes L \otimes \mathcal{O}_X / \mathcal{J}_Y^k) \). By the induction hypothesis, we find \( F^{(k-1)} \in H^0(X, K_X \otimes L) \) such that \( J^{k-1} F^{(k-1)} = J^{k-1} f \). 

Now consider \( f - J^k F^{(k-1)} \). This difference has a vanishing \((k-1)\)-th jet, i.e., it lies in the kernel of the projection to \( \mathcal{O}_X / \mathcal{J}_Y^k \). Therefore, it can be viewed as a section of \( S^k N_{Y/X}^* \) via the injection in the short exact sequence.

 \textbf{3. Identifying the Sheaf:}
Since \( N_{Y/X} \) is the normal bundle, its symmetric power \( S^k N_{Y/X}^* \) describes jets of order \( k \). Moreover, as \( N_{Y/X} \cong E_{\mid Y} \), we identify \( S^k N_{Y/X}^* \) with \( S^k E_{\mid Y}^* \). Thus, \( f - J^k F^{(k-1)} \) becomes a holomorphic section of the sheaf \( K_X \otimes L \otimes S^k E_{\mid Y}^* \).

 \textbf{Conclusion:}
The structure of the exact sequence allows us to interpret the jet \( f - J^k F^{(k-1)} \) as a section of \( K_X \otimes L \otimes S^k N_{Y/X}^* = K_X \otimes L \otimes S^k E_{\mid Y}^* \).}.

Using the results in \cite[p12]{Popovici2004L2EF}, one can construct an extension $\tilde{f} \in C^{\infty}\left(X, K_X \otimes L\right)$ of the holomorphic $k$-jet $f \in H^0\left(X, K_X \otimes L\right.$\\ $\left. \otimes \mathcal{O}_X / \mathcal{J}_Y^{k+1}\right)$ by means of a partition of unity, satisfying
$$
\bar{\partial} \tilde{f}=0 \quad \text { on } Y ,
$$
and
$$
|\bar{\partial} \tilde{f}|=O\left(|s|^{k+1}\right) \quad \text { in a neighbourhood of } Y .
$$
% ------------------------------- Clear sidenotes ------------------------------ %
% \softclearmargin
% ---------------------------------------------------------------------------- %
{\color{purple} Set\sidenote{Since we hardly know
$\tilde{f}$ away from $Y$ , we take a truncation with support in a tubular neighbourhood of $Y$.}
$$
G_{\varepsilon}^{(k-1)}:=\theta\left(\frac{\varepsilon^2}{|s|^2+\varepsilon^2}\right)\left(\tilde{f}-F^{(k-1)}\right) \in C^{\infty}\left(X, K_X \otimes L\right),
$$}
where $0<\varepsilon<\varepsilon_a$, and $\theta: \mathbb{R} \rightarrow[0,1]$ is a $C^{\infty}$ function such that $\theta \equiv 0$ on $(-\infty, \frac{l}{2}]$; $\theta \equiv 1$ on $\left[1-\frac{l}{2},+\infty\right)$, and $\left|\theta^{\prime}\right| \leq \frac{1+l}{1-l}$ on $\mathbb{R}$. {\color{purple} Then it suffices to solve the equation\sidenote{Here $u_\varepsilon$ is used to construct the erro term between $F^{(k)}$ and $F^{(k-1)}$. Later, we will see that it is the key to end the proof.}
\begin{equation}\label{eq:target}
  \bar{\partial} u_{\varepsilon}=\bar{\partial} G_{\varepsilon}^{(k-1)},
\end{equation}
with the extra condition $\frac{\left|u_{\varepsilon}\right|^2}{ |s|^{2m}} \in L_{\text {loc }}^1$ in a neighbourhood of $Y$. This condition guarantees that $u_{\varepsilon}$, as well as all its jets of orders $\leq k$, vanishes on $Y$.}
By direct calculations, one has
$$
\bar{\partial} G_{\varepsilon}^{(k-1)}=g_{\varepsilon}^{(1)}+g_{\varepsilon}^{(2)},
$$
where
$$
\begin{aligned}
& g_{\varepsilon}^{(1)}=-\frac{\varepsilon^2}{|s|^2+\varepsilon^2} \cdot \theta^{\prime}\left(\frac{\varepsilon^2}{|s|^2+\varepsilon^2}\right) \bar{v}_{\varepsilon} \wedge\left(\tilde{f}-F^{(k-1)}\right), \\
& g_{\varepsilon}^{(2)}=\theta\left(\frac{\varepsilon^2}{|s|^2+\varepsilon^2}\right) \bar{\partial}\left(\tilde{f}-F^{(k-1)}\right) .
\end{aligned}
$$
Recall that $v_{\varepsilon}$ is given in \eqref{eq:3}.


In this situation, $g_{\varepsilon}^{(2)}$ turns out to have no contribution in the limit since it converges uniformly to 0 on every compact set when $\varepsilon$ tends to 0 . Actually, Supp $\left(g_{\varepsilon}^{(2)}\right) \subset\{|s|<\sqrt{2} \varepsilon\}$ and $\left|g_{\varepsilon}^{(2)}\right|=O\left(|s|^{k+1}\right)$ because of $|\bar{\partial} \widetilde{f}|=O\left(|s|^{k+1}\right)$ in a neighbourhood of $Y$ as we have previously shown.


Then
$$
\int_{X_c \backslash Y}\left\langle B_{\varepsilon}^{-1} g_{\varepsilon}^{(2)}, g_{\varepsilon}^{(2)}\right\rangle_L|s|^{-2m} e^{-\psi}\mathrm{d} V_{X, \omega}=O(\varepsilon),
$$
provided that $B_{\varepsilon}$ is locally uniformly bounded below in a neighbourhood of $Y$. Otherwise, we shall solve the approximate equation $\bar{\partial} u+\sqrt{\delta} h=g_{\varepsilon}$ with $\delta>0$ small (see Lemma \ref{lem:L2existence} and \cite[Remark 3.2]{Demailly2000} for more details). One can remove the extra error term $\sqrt{\delta} h$ by putting $\delta \rightarrow 0$ at the end. Since there is no essential difficulty during this procedure, for the purpose of simplicity, we will assume to have the desired lower bound for $B_{\varepsilon}$ and the estimate of $g_{\varepsilon}^{(2)}$ as above.

Next, we turn to estimate the term involving $g_{\varepsilon}^{(1)}$ on $X_c \backslash Y$. By \eqref{eq:9},
{\color{purple} $$
\left\langle B_{\varepsilon}^{-1} g_{\varepsilon}^{(1)}, g_{\varepsilon}^{(1)}\right\rangle_{L_{a, \varepsilon}} \leq \frac{|s|^2}{m \varepsilon^2} \cdot\left|\theta^{\prime}\left(\frac{\varepsilon^2}{|s|^2+\varepsilon^2}\right) \frac{\varepsilon^2}{|s|^2+\varepsilon^2}\left(\tilde{f}-F^{(k-1)}\right)\right|_{L_{a, \varepsilon}}^2 .
$$}

In \cite[p17]{Popovici2004L2EF}, Popovici showed that on every compact set, 
\[
\frac{\left|\left(\tilde{f}-F^{(k-1)}\right)\left(\varepsilon s, z^{\prime}\right)\right|_L ^2}{\varepsilon^{2 k}} \longrightarrow  \left|\nabla^k\left(f-J^k F^{(k-1)}\right)\left(z^{\prime}\right)\right|_L^2 ,\quad (\varepsilon\to 0).
\]
Then using a partition of unity $\{\xi_p\}_{p=1}^{p_0}$  around $\overline{X_c} \backslash Y$ and the Fubini theorem, we obtain
\begin{fulleq}
 \begin{align*}
      \int_{X_c \backslash Y}\left\langle\mathrm{~B}_{\varepsilon}^{-1} g_{\varepsilon}^{(1)}, g_{\varepsilon}^{(1)}\right\rangle_{L_{a, \varepsilon}} e^{-\psi}\dd V_X 
     & \leq \frac{e^a(1+l)^2}{m(1-l)^2} \sum_{p=1}^{p_0} \int_{X_c \cap\left\{\sqrt{\frac{l}{2-l}} \varepsilon<|s|<\sqrt{\frac{2-l}{l}} \varepsilon\right\}} \frac{\varepsilon^2 \xi_p |\tilde{f}-F^{(k-1)} |^2_{L} e^{-\psi}\dd V_X}{\left(|s|^2+\varepsilon^2\right)^2|s|^{2 m-2}} \\
     & \leq \sum_{p=1}^{p_0}\left(\int_{\left\{z \in \mathbb{C}^m: \sqrt{\frac{l}{2-l}} \varepsilon<|z|<\sqrt{\frac{2-l}{l}} \varepsilon\right\}} \frac{\varepsilon^2(\sqrt{-1})^{m^2} \bigwedge^m(d z) \wedge \bigwedge^m(d \bar{z})}{\left(|z|^2+\varepsilon^2\right)^2|z|^{2 m-2}}\right. \\
     & \left.\times \frac{e^a(1+l)^2}{m(1-l)^2} \int_{Y_c} \frac{\xi_p \left|\nabla^k \left(\tilde{f}-J^k F^{(k-1)}\right)\right| ^2_{L} }{\left|\wedge^m(d s)\right|^2} e^{-\psi}\dd V_{Y,\omega}\right) \\
 &\rightarrow \frac{e^a(1+l)^2}{m(1-l)^2} C_{m,k}\int_{Y_c} \frac{ \left|\nabla^k \left(f-J^k F^{(k-1)}\right) \right|^2_{L} }{\left|\wedge^m(d s)\right|^2} e^{-\psi}\dd V_{Y,\omega}\quad (\varepsilon\to 0),
 \end{align*}
\end{fulleq}
where 
\begin{equation*}
  C_{m,k}:=\int_{\left\{z \in \mathbb{C}^m: \sqrt{\frac{l}{2-l}} <|z|<\sqrt{\frac{2-l}{l}} \varepsilon\right\}} \frac{(\sqrt{-1})^{m^2} \bigwedge^m(d z) \wedge \bigwedge^m(d \bar{z})}{\left(|z|^2+1\right)^2|z|^{2 m-2}},
\end{equation*}
% $$
% \begin{aligned}
% & \int_{X_c \backslash Y}\left\langle B_{\varepsilon}^{-1} g_{\varepsilon}^{(1)}, g_{\varepsilon}^{(1)}\right\rangle_{L_{a, \varepsilon}} \mathrm{d} V_X \leq \frac{16 e^a}{m} \int_{X_c \cap\left\{\sqrt{\frac{1}{2}} \varepsilon<|s|<\sqrt{2} \varepsilon\right\}} \frac{\varepsilon^2\left|\tilde{f}-F^{(k-1)}\right|_L^2 \mathrm{~d} V_X}{\left(|s|^2+\varepsilon^2\right)^2|s|^{2(m+k-1)}} \\
% & \longrightarrow \frac{16 e^a}{m} C_{m, k} \int_{Y_c} \frac{\left|\nabla^k\left(f-J^k F^{(k-1)}\right)\right|_L^2}{\left|\Lambda^m(\mathrm{~d} s)\right|^{2 \frac{m+k}{m}}} \mathrm{~d} V_{Y, \omega} \quad(\varepsilon \longrightarrow 0),
% \end{aligned}
% $$
which depends only on $m$ and $k$. It may be worthwhile to note that $$\left|\nabla^k\left(f-J^k F^{(k-1)}\right)\right|_L=\left|f-J^k F^{(k-1)}\right|_L,$$ where $f-J^k F^{(k-1)} \in H^0\left(Y, K_X \otimes L \otimes S^k N_{Y / X}^*\right)$. Then, one has
 \begin{fullwidth}
 $$
 \int_{X_c \backslash Y}\left\langle B_{\varepsilon}^{-1} g_{\varepsilon}^{(1)}, g_{\varepsilon}^{(1)}\right\rangle_{L_{a, \varepsilon}} e^{-\psi}\mathrm{d} V_X \leq \frac{e^a(1+l)^2}{m(1-l)^2} C_{m, k} \int_{Y_c} \frac{\left|\nabla^k\left(f-J^k F^{(k-1)}\right)\right|_{L}^2}{\left|\Lambda^m(\mathrm{~d} s)\right|^{2 }} e^{-\psi}\mathrm{~d} V_{Y, \omega},
 $$
\end{fullwidth}
when $\varepsilon$ is small enough. By using Lemma \ref{lem:L2existence}\sidenote[][0pt]{$L^2$ existence theorem with erro term $\delta=0$.} with $\delta=0$, we can solve \eqref{eq:target}, i.e., there exists $u_{c, a, \varepsilon} \in L^2\left(X_c \backslash Y, K_X \otimes L_{a, \varepsilon}\right)$ such that
$$
\bar{\partial} u_{c, a, \varepsilon}=\bar{\partial} G_{\varepsilon}^{(k-1)}=g_{\varepsilon}^{(1)}+g_{\varepsilon}^{(2)}
$$
on $X_c \backslash Y$ and

 \begin{fullwidth}
 \begin{equation}\label{eq:12}
   \int_{X_c \backslash Y} \frac{\left|u_{c, a, \varepsilon}\right|_{L}^2 e^{-\sigma-\zeta\left(\sigma_{\varepsilon}\right)}}{\tau_{\varepsilon}+A_{\varepsilon}} e^{-\psi}\mathrm{d} V_X \leq \frac{e^a(1+l)^2}{m(1-l)^2} C_{m, k} \int_{Y_c} \frac{\left|\nabla^k\left(f-J^k F^{(k-1)}\right)\right|_{L}^2}{\left|\Lambda^m(\mathrm{~d} s)\right|^{2 }} e^{-\psi}\mathrm{~d} V_{Y, \omega}+O(\varepsilon) .
 \end{equation}
\end{fullwidth}


Since $\sigma, \zeta\left(\sigma_{\varepsilon}\right), \tau_{\varepsilon}+A_{\varepsilon}$ are all bounded above on $\overline{X_c}$ for each fixed $\varepsilon$, the inequality \eqref{eq:12} implies that $u_{c, a, \varepsilon} \in L^2\left(X_c, K_X \otimes L\right)$. As \eqref{eq:target}, \eqref{eq:12} and $G_{\varepsilon}^{k-1}$ is smooth, Lemma \ref{lem:6}\sidenote[][-50pt]{\begin{minipage}{\marginparwidth}
 \begin{lemma}[Lemma 6.9 in \cite{Demailly1982}]
   Let $\Omega$ be an open subset of $\mathbb{C}^n$ and $Z$ be a complex analytic subset of $\Omega$. Assume that $v$ is a $(p, q-1)$-form with $L_{\text {loc }}^2$ coefficients and $h$ is a $(p, q)$-form with $L_{\textrm{loc}}^1$ coefficients such that $\bar{\partial} v=h$ on $\Omega \backslash Z$ (in the sense of distribution theory). Then $\bar{\partial} v=h$ on $\Omega$.
 \end{lemma}
\end{minipage}} gives that
\begin{equation}\label{eq:13}
  \bar{\partial} u_{c, a, \varepsilon}=\bar{\partial} G_{\varepsilon}^{(k-1)}=g_{\varepsilon}^{(1)}+g_{\varepsilon}^{(2)}
\end{equation}
extends across $Y$ and
 \begin{fulleq}
 \begin{equation}\label{eq:14}
   \int_{X_c} \frac{\left|u_{c, a, \varepsilon}\right|_{L}^2 e^{-\sigma-\zeta\left(\sigma_{\varepsilon}\right)}}{\tau_{\varepsilon}+A_{\varepsilon}} e^{-\psi}\mathrm{d} V_X \leq \frac{e^a(1+l)^2}{m(1-l)^2}C_{m, k} \int_{Y_c} \frac{\left|\nabla^k\left(f-J^k F^{(k-1)}\right)\right|_{L}^2}{\left|\Lambda^m(\mathrm{~d} s)\right|^{2 }} e^{-\psi}\mathrm{~d} V_{Y, \omega}+O(\varepsilon) .
 \end{equation}
\end{fulleq}

{\color{purple} The $k$-jet extension\sidenote{\small Here is the key point of the proof. The $k$-th term is represented by the $k-1$ term, which is also the core step of the induction. In above we set \[G_{\varepsilon}^{(k-1)}:=\theta\left(\frac{\varepsilon^2}{|s|^2+\varepsilon^2}\right)\left(\tilde{f}-F^{(k-1)}\right) ,\]
and we have known that \[
 \begin{aligned}
 \bdd u_\varepsilon&=\bdd G_\varepsilon^{(k-1)}\sim g_\varepsilon^{(1)}\\
 &=-\frac{\varepsilon^2}{|s|^2+\varepsilon^2} \cdot \theta^{\prime}\left(\frac{\varepsilon^2}{|s|^2+\varepsilon^2}\right) \bar{v}_{\varepsilon} \\
 &\wedge\left(\tilde{f}-F^{(k-1)}\right).
\end{aligned}
\]
% Thus by \ref{lem:L2existence} there exists a solution $u_{c,a,\varepsilon}$ satisfying \ref{eq:13} and \ref{eq:14}, and 
} of $f$ to $X_c$ is then given by
$$
F_{c, a, \varepsilon}^{(k)}:=\br{G_{\varepsilon}^{(k-1)}-u_{c, a, \varepsilon}}+F^{(k-1)} .
$$}

Then $F_{c, a, \varepsilon}^{(k)}$ is holomorphic on $X_c$, thanks to \eqref{eq:13}, $F^{(k-1)} \in H^0\left(X, K_X \otimes L\right)$ as well as the ellipticity of the operator $\bar{\partial}$ in bidegree $(n, 0)$. So $u_{c, a, \varepsilon}$ is also smooth on $X_c$. Locally, near an arbitrary point of $Y$, all partial derivatives of order $s\leq k$ of $F_{c, a, \varepsilon}^{(k)}$ are prescribed by $f$.


By the variant of Cauchy-Schwarz inequality, we have
 \begin{equation}\label{eq:15}
   \begin{aligned}
   &\left\langle\kappa_1+\kappa_2+\kappa_3, \kappa_1+\kappa_2+\kappa_3\right\rangle \\
   & \leq (1+l)\left\langle\kappa_1+\kappa_2, \kappa_1+\kappa_2\right\rangle+(1+\frac{1}{l})\left\langle\kappa_3, \kappa_3\right\rangle \\
   & \leq (1+l)^2 \left\langle\kappa_1, \kappa_1\right\rangle+\frac{(1+l)^2}{l}\left\langle\kappa_2, \kappa_2\right\rangle+(1+\frac{1}{l})\left\langle\kappa_3, \kappa_3\right\rangle
   \end{aligned}
 \end{equation}
for any inner product space $(\mathcal{H},\langle\cdot, \cdot\rangle), \kappa_1, \kappa_2, \kappa_3 \in \mathcal{H}$.

Then for some sufficiently small $\varepsilon, R\left(\sigma_{\varepsilon}\right) \leq R(\sigma), R\left(m \log |s|^2\right) \leq R(\sigma)$, the induction hypothesis \eqref{eq:10}, \eqref{eq:14} and \eqref{eq:15} give the estimate on the relatively compact open subset $X_c$,
\clearpage
   \begin{fullwidth}
   \begin{align}\label{eq:16}
   & \int_{X_c} \frac{\left|F_{c, a, \varepsilon}^{(k)}\right|_{L}^2}{e^{\sigma} R(\sigma)} e^{-\psi}\mathrm{d} V_X  \notag\\
   & \leq (1+l)^2 \int_{X_c} \frac{\left|u_{c, a, \varepsilon}\right|_{L}^2}{e^{\sigma} R(\sigma)} e^{-\psi}\mathrm{d} V_X+\frac{(1+l)^2}{l} \int_{X_c} \frac{\left|F^{(k-1)}\right|_{L}^2}{e^{\sigma} R(\sigma)} e^{-\psi}\mathrm{d} V_X+(1+\frac{1}{l}) \int_{X_c} \frac{\left|G_{\varepsilon}^{(k-1)}\right|_{L}^2}{e^{\sigma} R(\sigma)} e^{-\psi}\mathrm{d} V_X  \notag\\
   & \leq (1+l)^2\left(\sup _{X_c} \frac{\left(\tau_{\varepsilon}+A_{\varepsilon}\right) e^{\zeta\left(\sigma_{\varepsilon}\right)}}{R\left(\sigma_{\varepsilon}\right)}\right) \int_{X_c} \frac{\left|u_{c, a, \varepsilon}\right|_{L}^2 e^{-\sigma-\zeta\left(\sigma_{\varepsilon}\right)}}{\tau_{\varepsilon}+A_{\varepsilon}} e^{-\psi}\mathrm{d} V_X  \notag\\
   & +\frac{(1+l)^2}{l}  e^{ a} \int_{X_c} \frac{\left|F^{(k-1)}\right|_{L}^2}{|s|^{2 m} R\left(m \log |s|^2\right)} e^{-\psi}\mathrm{d} V_X+(1+\frac{1}{l}) e^{a} \int_{X_c} \frac{\theta\left(\frac{\varepsilon^2}{|s|^2+\varepsilon^2}\right)^2\left|\tilde{f}-F^{(k-1)}\right|_{L}^2}{|s|^{2 m} R\left(m \log |s|^2\right)} e^{-\psi}\mathrm{d} V_X  \notag\\
   & \leq (1+l)^2 \left(\sup _{X_c} \frac{\left(\tau_{\varepsilon}+A_{\varepsilon}\right) e^{\zeta\left(\sigma_{\varepsilon}\right)}}{R\left(\sigma_{\varepsilon}\right)}\right) \int_{X_c} \frac{\left|u_{c, a, \varepsilon}\right|_{L}^2 e^{-\sigma-\zeta\left(\sigma_{\varepsilon}\right)}}{\tau_{\varepsilon}+A_{\varepsilon}} e^{-\psi}\mathrm{d} V_X  \notag\\
   & +\frac{(1+l)^2}{l}  e^{a} \int_{X_c} \frac{\left|F^{(k-1)}\right|_{L}^2}{|s|^{2 m} R\left(m \log |s|^2\right)} e^{-\psi}\mathrm{d} V_X+(1+\frac{1}{l}) C_1 e^{a} \int_{X_c} \frac{1}{|s|^{2 m} R\left(m \log |s|^2\right)} e^{-\psi}\mathrm{d} V_X  \notag\\
   & \leq \frac{(1+l)^4 e^a}{m (1-l)^2}\left(\sup _{X_c} \frac{\left(\tau_{\varepsilon}+A_{\varepsilon}\right) e^{\zeta\left(\sigma_{\varepsilon}\right)}}{R\left(\sigma_{\varepsilon}\right)}\right) C_{m, k} \int_{Y_c} \frac{\left|\nabla^k\left(f-J^k F^{(k-1)}\right)\right|_{L}^2}{\left|\bigwedge^m(\mathrm{~d} s)\right|^{2 }} e^{-\psi}\mathrm{~d} V_Y  \notag\\
   & +\frac{(1+l)^2}{l}  e^{a} C_{m, R}^{(k-1)} \int_{Y_c} \frac{|f|_{s, \rho,(k-1)}^2}{\left|\bigwedge^m(\mathrm{~d} s)\right|^2} e^{-\psi}\mathrm{~d} V_Y+C_2 \int_{-\infty}^{2 m \log \varepsilon+C_3} \frac{1}{R(t)} \mathrm{d} t+O(\varepsilon)  \notag\\
   & \leq e^{a} C_{m, R}^{\prime(k)} \int_{Y_c} \frac{|f|_{s, \rho,(k)}^2}{\left|\Lambda^m(\mathrm{~d} s)\right|^2} e^{-\psi}\mathrm{~d} V_Y  +\frac{(1+l)^4 e^a}{m (1-l)^2} C_{m, k} \int_{Y_c} \frac{\left|\nabla^k\left(J^k F^{(k-1)}\right)\right|_{L}^2}{\left|\bigwedge^m(\mathrm{~d} s)\right|^{2 }} e^{-\psi}\mathrm{~d} V_Y \notag\\
   &+C_2 \int_{-\infty}^{2 m \log \varepsilon+C_3} \frac{1}{R(t)} \mathrm{d} t+O(\varepsilon), 
   \end{align}
  \end{fullwidth}
where $C_{m, R}^{\prime(k)}=\frac{(1+l)^4 }{m (1-l)^2} C_{m, k}+\frac{(1+l)^2}{l} C_{m, R}^{(k-1)}$ and $C_1, C_2, C_3$ are all positive numbers independent of $\varepsilon$. Here in (16), we also assume that
\begin{equation}\label{eq:17}
  \frac{\left(\tau_{\varepsilon}+A_{\varepsilon}\right) e^{\zeta\left(\sigma_{\varepsilon}\right)}}{R\left(\sigma_{\varepsilon}\right)}=1
\end{equation}
on $X_c$. We will solve \eqref{eq:17} together with \eqref{eq:5} and \eqref{eq:8} in above. 

\subsection{Final jet extensions on \texorpdfstring{$\Omega$}{} via limits}

As $\sup _{t \leq 0}\left(e^t R(t)\right)<\infty$, applying Montel's theorem and \eqref{eq:16} to extract a weak limit of $\left\{F_{c, a, \varepsilon}^{(k)}\right\}_{\varepsilon>0}$ {\color{purple} as $\varepsilon \rightarrow 0$\sidenote[][0pt]{First Limitation.}}, we get a holomorphic $L$-valued $n$-form $F_{c, a}^{(k)}$ on $X_c$ such that $J_{X_c}^k F_{c, a}^{(k)}=f$ and
 \begin{fullwidth}
 $$
 \int_{X_c} \frac{\left|F_{c,a}^{(k)}\right|_L^2}{e^{\sigma} R(\sigma)} e^{-\psi}\mathrm{d} V_X \leq e^{ a} C_{m, R}^{\prime(k)} \int_{Y_c} \frac{|f|_{s, \rho,(k)}^2}{\left|\bigwedge^m(\mathrm{~d} s)\right|^2} e^{-\psi}\mathrm{~d} V_Y+\frac{(1+l)^4 e^a}{m (1-l)^2} C_{m, k} \int_{Y_c} \frac{\left|\nabla^k\left(J^k F^{(k-1)}\right)\right|_L^2}{\left|\bigwedge^m(\mathrm{~d} s)\right|^{2 }} e^{-\psi}\mathrm{~d} V_Y .
 $$
\end{fullwidth}

In other words,
 \begin{fullwidth}
 \begin{equation}\label{eq:18}
   \begin{aligned}
   \int_{X_c} \frac{\left|F_{c, a}^{(k)}\right|_L^2 e^{-\psi}\mathrm{d} V_X}{|s|^{2 m} R\left(m \log |s|^2-a\right)}   
   & \leq C_{m, R}^{\prime(k)} \int_{Y_c} \frac{|f|_{s, \rho,(k)}^2 e^{-\psi}\mathrm{~d} V_Y}{\left|\bigwedge^m(\mathrm{~d} s)\right|^2} +\frac{(1+l)^4 C_{m, k} }{m (1-l)^2} \int_{Y_c} \frac{\left|\nabla^k\left(J^k F^{(k-1)}\right)\right|_L^2 e^{-\psi}\mathrm{~d} V_Y}{\left|\bigwedge^m(\mathrm{~d} s)\right|^{2}} 
   \end{aligned}
 \end{equation}
\end{fullwidth}

Since $R$ is continuous decreasing on $(-\infty, 0], \sup _{t \leq 0}\left(e^t R(t)\right)<\infty$, similarly as before, we use Montel's theorem and extract a weak limit of $\left\{F_{c, a}^{(k)}\right\}_{a>0}$ {\color{purple} as $a \rightarrow 0$\sidenote[][0pt]{Second Limitation.}}, to obtain a holomorphic $L$-valued $n$-form $F_c^{(k)}$ on $X_c$ from \eqref{eq:18} such that $J_{X_c}^k F_c^{(k)}=f$ and
 \begin{fullwidth}
 \begin{equation}\label{eq:19}
   \int_{X_c} \frac{\left|F_{c}^{(k)}\right|_{L}^2 e^{-\psi}}{|s|^{2 m} R\left(m \log |s|^2\right)} \mathrm{d} V_X \leq C_{m, R}^{\prime(k)} \int_{Y_c} \frac{|f|_{s, \rho,(k)}^2 e^{-\psi}}{\left|\bigwedge^m(\mathrm{~d} s)\right|^2} \mathrm{~d} V_Y+\frac{(1+l)^4 C_{m, k}}{m (1-l)^2}  \int_{Y_c} \frac{\left|\nabla^k\left(J^k F^{(k-1)}\right)\right|_L^2}{\left|\bigwedge^m(\mathrm{~d} s)\right|^2 } e^{-\psi}\mathrm{~d} V_Y .
 \end{equation}
\end{fullwidth}

As Popovici \cite[Sections 0.4-0.6]{Popovici2004L2EF} has shown that the last term in the right-hand side of \eqref{eq:19} can be controlled uniformly, a slight modification of his proof in \cite[Section 0.4]{Popovici2004L2EF}  in terms of the variable denominators introduced by \cite[p136]{ZZ18} can complete the proof of Theorem \ref{thm:L2}. Indeed, one just needs to modify the first and second inequalities in \cite[p22]{Popovici2004L2EF}, respectively, as
 \begin{fullwidth}
 $$
 \begin{aligned}
 & \frac{\sum_{|\alpha|=k}\left|\frac{\frac{\partial^\alpha F^{(k-1)}}{\partial z^{\prime \alpha}}\left(0, z^{\prime \prime}\right)}{\alpha!}\right|^2 e^{-2 \varphi\left(0, z^{\prime \prime}\right)-2 A\left|z^{\prime \prime}\right|^2}}{\left|\bigwedge^m(\mathrm{~d} s)\left(0, z^{\prime \prime}\right)\right|^{2 \frac{(m+k)}{m}}} \\
 & \leq \text { Const } \cdot \frac{2(m+k)}{\rho^{2(m+k)}} e^{2\left(\varepsilon(\rho)+A \rho^2\right)} \sup _{\left(z^{\prime}, z^{\prime \prime}\right) \in U_j} \frac{\left|s\left(z^{\prime}, z^{\prime \prime}\right)\right|^{2 m} R\left(m \log s\left(z^{\prime}, z^{\prime \prime}\right)^2\right)}{\left|\bigwedge^m(\mathrm{~d} s)\left(0, z^{\prime \prime}\right)\right|^{2 \frac{(m+k)}{m}}} \\
 & \times \int_{z^{\prime} \in B^{\prime}(0, \rho)} \frac{\left\|F^{(k-1)}\left(z^{\prime}, z^{\prime \prime}\right)\right\|^2}{\left|s\left(z^{\prime}, z^{\prime \prime}\right)\right|^{2 m} R\left(m \log s\left(z^{\prime}, z^{\prime \prime}\right)^2\right)} \mathrm{d} \lambda\left(z^{\prime}\right), 
 \end{aligned}
 $$
\end{fullwidth}
and
 \begin{fullwidth}
 $$
 \int_{Y_c} \frac{\left|\nabla^k\left(J^k F^{(k-1)}\right)\right|^2}{\left|\bigwedge^m(\mathrm{~d} s)\right|^{2 }} e^{-\psi}\mathrm{~d} V_Y \leq D_{m, k} N M(c) \frac{1}{\rho^{2(m+k)}} e^{2\left(\varepsilon(\rho)+A \rho^2\right)} \int_{\Omega^{\prime}} \frac{\left\|F^{(k-1)}\right\|^2}{|s|^{2 m} R\left(m \log |s|^2\right)} e^{-\psi}\mathrm{d} V_{X, \omega},
 $$
\end{fullwidth}
where
$$
M(c):=\sup _{\left(z^{\prime}, z^{\prime \prime}\right) \in \Omega^{\prime}} \frac{\left|s\left(z^{\prime}, z^{\prime \prime}\right)\right|^{2 m} R\left(m \log s\left(z^{\prime}, z^{\prime \prime}\right)^2\right)}{\left|\bigwedge^m(\mathrm{~d} s)\left(0, z^{\prime \prime}\right)\right|^{2 \frac{m+k}{m}}} .
$$

and $D_{m, k}:=$ Const $\cdot 2(m+k)$. Notice that the smoothness of the function $R$ on $(-\infty, 0]$ ensures that one can get the suprema on $U_j$ and $\Omega^{\prime}$, respectively. We refer to \cite[Section 0.4]{Popovici2004L2EF} for more explanations about the above notations.
Then as a result, we get a holomorphic $L$-valued $n$-form $F_{c}^{(k)}$ on $\Omega$ such that $J_{\Omega}^k F_{c}^{(k)}=f$ and
 \begin{fulleq}
 $$
 \begin{aligned}
 \int_{\Omega} \frac{\left|F_{c}^{(k)}\right|_{L}^2 e^{-\psi}}{|s|_E^{2 m} R\left(m \log |s|_E^2\right)} \mathrm{d} V_{X, \omega}  \leq \int_{X_c} \frac{\left|F_{c}^{(k)}\right|_{L}^2 e^{-\psi}}{|s|_E^{2 m} R\left(m \log |s|_E^2\right)} \mathrm{d} V_{X, \omega} 
 & \leq C_{m, R}^{(k)} \int_{Y_c} \frac{|f|_{s, \rho,(k)}^2}{\left|\bigwedge^m(\mathrm{~d} s)\right|_E^2} e^{-\psi}\mathrm{~d} V_{Y, \omega} \\
 & \leq C_{m, R}^{(k)} \int_Y \frac{|f|_{s, \rho,(k)}^2}{\left|\bigwedge^m(\mathrm{~d} s)\right|_E^2}  e^{-\psi}\mathrm{~d} V_{Y, \omega},
 \end{aligned}
 $$
\end{fulleq}
where $C_{m, R}^{(k)}>0$ is a constant depending only on $m, k, E, R$ and $\sup _{\Omega}\|i \Theta(L)\|$.

\subsection{Solving ordinary differential equations}

We have already proved Theorem \ref{thm:main} , provided that there exist appropriate $\chi, \eta, \zeta$ satisfying some assumptions. Now, we will come to use these assumptions about $\chi, \eta, \zeta$ to get their explicit expressions.

Notice that \eqref{eq:5}, \eqref{eq:8} and \eqref{eq:17} are equivalent to the following system of ordinary differential equations defined on $(-\infty, 0)$ :
$$
\left\{
\begin{aligned}
\chi(t) \zeta^{\prime}(t)-\chi^{\prime}(t) &=1, \\
(\chi(t)+\eta(t)) e^{\zeta(t)} &=R( t), \\
\frac{\left(\chi^{\prime}(t)\right)^2}{\chi(t) \zeta^{\prime \prime}(t)-\chi^{\prime \prime}(t)} &=\eta(t).
\end{aligned}\right.
$$
 Moreover, we have assumed that $\zeta, \chi$ and $\eta$ are all smooth on $(-\infty, 0)$ and that $\zeta>0, \chi>0, \eta>0, \zeta^{\prime}>0, \chi^{\prime}<0$ and $\chi(t) \geq-\frac{t}{2}$ on $(-\infty, 0)$. In the proof of Theorem \ref{thm:main}, we have assumed that $C_R=\int_{-\infty}^0 \frac{1}{R(t)} \mathrm{d} t=1$. 

Following the argument of solving undetermined functions with ODEs introduced in \cite[Section 4, pp. 151-153]{ZZ18}, we get
$$
\left\{
\begin{aligned}
\zeta &=-\log \left(1-\int_{-\infty}^t \frac{1}{R\left( t_1\right)} \mathrm{d} t_1\right), \\
\chi &=\frac{- t- \int_t^0\left(\int_{-\infty}^{t_2} \frac{1}{R\left( t_1\right)} \mathrm{d} t_1\right) \mathrm{d} t_2}{1-\int_{-\infty}^t \frac{1}{R\left( t_1\right)} \mathrm{d} t_1}, \\
\eta &=\left(1- \int_{-\infty}^t \frac{1}{R\left( t_1\right)} \mathrm{d} t_1\right) R( t)+\frac{ t+\int_t^0\left(\int_{-\infty}^{t_2} \frac{1}{R\left( t_1\right)} \mathrm{d} t_1\right) \mathrm{d} t_2}{1- \int_{-\infty}^t \frac{1}{R\left( t_1\right)} \mathrm{d} t_1},
\end{aligned}\right.
$$
and
$$
\chi^{\prime}+\frac{1}{2}=\left(\frac{-\frac{1}{2}\left(\lambda_1^{\prime}\right)^2+\lambda_1 \lambda_1^{\prime \prime}}{\left(\lambda_1^{\prime}\right)^2}\right) \leq 0 .
$$

It is easy to verify all the previous assumptions about $\zeta, \chi$ and $\eta$.
In the end, we have proven the $L^2$-extension  theorem \ref{thm:L2}.
\end{proof}

Now we will show that the main theorem which is the case of $L^q$-extension is true later.

\section{Proof of the main theorem}
From now on, we will denote $F^{(k)}$ in theorem \ref{thm:L2} by $F_1^{(k)}$ .
$K_X$ is naturally equipped with the smooth metric $e^{\varphi_\omega}$ with respect to the dual frame of $d z$. Let $L^{\prime}$ be the line bundle $L$ equipped with the new metric $e^{-\varphi_{L^{\prime}}}$, where $\varphi_{L^{\prime}}:=(2-q) \log \left|F_1\right|_L+\varphi_L$. Then the assumptions in the theorem imply that
\begin{enumerate}[label=(\roman*)]
  \item $\sqrt{-1} \Theta_{L^{\prime}}+\sqrt{-1} \partial \bar{\partial} \sigma \geq 0$,
  \item $\sqrt{-1} \Theta_{L^{\prime}}+\sqrt{-1} \partial \bar{\partial} \sigma \geq \frac{\left\{\sqrt{-1} \Theta_E s, s\right\}_E}{\alpha|s|_E^2}$.
\end{enumerate}


Since the $k$-jet  $f \in H^0\left(X,K_X\otimes L'\otimes \mO_X/\mJ_Y^{k+1}\right)$ satisfies
$$
\int_Y \frac{|f|_{L^{\prime},s,\rho,(k)}^2 e^{-\psi}}{\left|\bigwedge^m(\dd s)\right|_E^2} d V_{Y,\omega}=C_f<+\infty,
$$
by Theorem \ref{thm:L2}, there exists  $F_2^{(k)}$ on $X$ with values in $K_X \otimes L'$, such that $J^k F_2^{(k)}=f$ on $\Omega$ and
$$
\begin{aligned}
&\int_\Omega \frac{\left|F_2^{(k)}\right|_{L}^2 e^{-\psi}}{\left(\left|F_1^{(k)}\right|_L\right)^{2-q}  |s|^{2m} R(\psi+m\log |s|^2)} d V_{X,\omega}  \\
=&\int_\Omega \frac{\left|F_2^{(k)}\right|_{L^{\prime}}^2 e^{-\psi}}{|s|^{2m} R(\psi+m\log |s|^2)} d V_{X,\omega}  
 \leq C_{m,R}^{(k)} \int_Y \frac{|f|_{L^{\prime},s,\rho,(k)}^2 e^{-\psi}}{\left|\bigwedge^m(\dd s)\right|_E^2} d V_{Y,\omega}  
 =C_{m,R}^{(k)} C_f .
\end{aligned}
$$

Then H\"older's inequality gives that
$$
\begin{aligned}
C_{F_2^{(k)}}  &:=\int_\Omega \frac{\left(\left|F_2^{(k)}\right|_{L}\right)^q   e^{-\psi}}{|s|^{2m} R(\psi+m\log |s|^2)} d V_{X,\omega} \\
& \leq\left(\int_\Omega \frac{\left|F_2^{(k)}\right|_{L}^2  e^{-\psi}}{\left(\left|F_1^{(k)}\right|_{L'}\right)^{2-q} |s|^{2m} R(\psi+m\log |s|^2)} d V_{X,\omega}\right)^{\frac{q}{2}}\left(\int_\Omega \frac{\left(\left|F_1^{(k)}\right|_{L}\right)^q e^{-\psi}}{|s|^{2m} R(\psi+m\log |s|^2)} d V_{X,\omega}\right)^{1-\frac{q}{2}} \\
& \leq\left(C_{m,R}^{(k)} C_f\right)^{\frac{q}{2}}\left(C_{F_1^{(k)}}\right)^{1-\frac{q}{2}} .
\end{aligned}
$$

We can then repeat the same argument with $F_1^{(k)}$ replaced by $F_2^{(k)}$, etc., and get a sequence of holomorphic extensions $\left\{F_s^{(k)}\right\}_{s=1}^{+\infty}$ of $f$ and a sequence $\left\{C_{F_s^{(k)}}\right\}_{s=1}^{+\infty}$ such that
\begin{equation}\label{eq:7.1}\color{purple}
  C_{F_{s+1}^{(k)}} \leq\left(C_{m,R}^{(k)} C_f\right)^{\frac{q}{2}}\left(C_{F_s^{(k)}}\right)^{1-\frac{q}{2}}, \quad s=1,2, \cdots .
\end{equation}
\begin{enumerate}[label=(\Roman*)]
  \item If $C_{F_s}^{(k)} \leq C_{m,R}^{(k)} C_f$ for some $C_{F_s}^{(k)}$, then we finish the proof since $F_s^{(k)}$ can be regarded as the desired $k$-jet  extension $F^{(k)}$ in the conclusion.
  \item If $C_{F_s^{(k)}}>C_{m,R}^{(k)} C_f$ for any $k$, then $C_{F_{s+1}^{(k)}}<C_{F_s^{(k)}}$ for any $s$. Since $\varphi_L$ is locally bounded above and $e^\sigma R(\sigma)$ is bounded above, applying Montel's theorem and extracting weak limits of $\left\{F_s^{(k)}\right\}_{s=1}^{+\infty}$, we can get from \eqref{eq:7.1} a $k$-jet $F^{(k)}$  on $X$ with values in $K_X \otimes L$, such that $J^k F^{(k)}=f$ on $\Omega$ and
$$
\int_\Omega \frac{\left(|F^{(k)}|_L\right)^q e^{-\psi}}{|s|^{2m} R(\psi+m\log |s|^2)} d V_X \leq C_{m,R}^{(k)}C_f .
$$
\end{enumerate}
Theorem \ref{thm:main} is, thus, proved.






\newgeometry{left=2cm,right=2cm,bottom=2cm,top=3cm}
%
\printbibliography[heading=bibintoc]
 %
\end{document}
